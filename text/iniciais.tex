Como uma primeira abordagem para a resolução do MCSP, apresentaremos duas heurísticas diretas e determinísticas, que utilizam alguma característica do problema para tentar encontrar soluções com boa qualidade: a heurística de combinação e a heurística gulosa.

\subsection{Heurística de Combinação}

    A heurística de combinação consiste em um algoritmo que inicia com uma solução trivial do problema (blocos de um caractere cada) e tenta agrupar blocos mantendo uma solução válida. A implementação escolhida consiste em analisar os blocos das strings da esquerda para a direita e agrupá-los sempre que possível, como mostra o \Cref{alg:combine}. A escolha arbitrária de quais blocos serão combinados, no entanto, compromete a qualidade da solução gerada.

    \begin{algorithm}[htb]
        \caption{Heurística de combinação.}
        \label{alg:combine}

        \begin{codebox}
        \Procname{$\proc{Combinação}(A, B)$}
        \li $B_A \Recebe$ blocos unitários de $A$
        \li $B_B \Recebe$ blocos unitários de $B$
        \li \Para \Cada par de blocos $(b_1, b_2)$ consecutivos em $B_A$ e em $B_B$ \Faca
            \Do
        \li     $B_A \Recebe B_A$ com $b_1$ e $b_2$ combinados
        \li     $B_B \Recebe B_B$ com $b_1$ e $b_2$ combinados
            \End
        \li \Devolva $(B_A, B_B)$
        \end{codebox}
    \end{algorithm}

    \subsubsection{Análise de Singletons}

        \begin{definition}[Singleton]
            Um caractere de uma string é dito singleton se a ocorrência de seu rótulo na string é 1.
        \end{definition}

        Caracteres singletons têm valor especial na resolução do MCSP: uma substring que contém tal caractere em $A$ não pode ter mais de uma substring equivalente em $B$. Com isso, é possível priorizar a combinação de blocos que contêm singletons e, com isso, melhorar a tomada de decisão do algoritmo e seus resultados. O \Cref{alg:combineS} mostra como adaptar o procedimento de combinação de blocos. Chamaremos essa modificação de combinação-S.

        \begin{algorithm}[htb]
        \caption{Heurística de combinação com análise de singletons.}
        \label{alg:combineS}

        \begin{codebox}
        \Procname{$\proc{Combinação-S}(A, B)$}
        \li $B_A \Recebe$ blocos unitários de $A$
        \li $B_B \Recebe$ blocos unitários de $B$
        \li \Para \Cada par de blocos $(b_1, b_2)$ consecutivos em $B_A$ e em $B_B$
        \zi tal que $b_1$ e $b_2$ possuam singletons \Faca
            \Do
        \li     $B_A \Recebe B_A$ com $b_1$ e $b_2$ combinados
        \li     $B_B \Recebe B_B$ com $b_1$ e $b_2$ combinados
            \End
        \li \Para \Cada par de blocos $(b_1, b_2)$ consecutivos em $B_A$ e em $B_B$
        \zi tal que $b_1$ ou $b_2$ possuam singletons \Faca
            \Do
        \li     $B_A \Recebe B_A$ com $b_1$ e $b_2$ combinados
        \li     $B_B \Recebe B_B$ com $b_1$ e $b_2$ combinados
            \End
        \li \Para \Cada par de blocos $(b_1, b_2)$ consecutivos em $B_A$ e em $B_B$ \Faca
            \Do
        \li     $B_A \Recebe B_A$ com $b_1$ e $b_2$ combinados
        \li     $B_B \Recebe B_B$ com $b_1$ e $b_2$ combinados
            \End
        \li \Devolva $(B_A, B_B)$
        \end{codebox}
    \end{algorithm}

\subsection{Heurística Gulosa}

    Uma outra estratégia para encontrar soluções do MCSP é a gulosa. Tal algoritmo consiste em iterativamente escolher blocos grandes que coincidam entre as duas strings e marcá-los. Em cada iteração, escolhe-se a maior substring comum entre $S$ e $P$ que não coincida com blocos já marcados em nenhuma das strings. Os dois novos blocos correspondentes são então marcados. O algoritmo progride dessa forma até que todos os caracteres das strings pertençam a blocos marcados, formando então uma partição comum.

    \begin{algorithm}[htb]
        \caption{Heurística gulosa.}
        \label{alg:greedy}

        \begin{codebox}
        \Procname{$\proc{Gulosa}(A, B)$}
        \li $B_A \Recebe$ sequência com $A$ como único elemento
        \li $B_B \Recebe$ sequência com $B$ como único elemento
        \li \Enquanto existem caracteres pertencentes a blocos não marcados \Faca
            \Do
        \li     $x \Recebe$ maior substring comum a $B_A$ e $B_B$ nos blocos não marcados
        \li     quebre um bloco de $B_A$ que contém $x$ criando um novo bloco marcado com apenas $x$
        \li     quebre um bloco de $B_B$ que contém $x$ criando um novo bloco marcado com apenas $x$
            \End
        \li \Devolva $(B_A, B_B)$
        \end{codebox}
    \end{algorithm}

    \subsubsection{Maior Substring Comum}

        O problema de encontrar a maior substring comum entre duas strings pode ser resolvido com a ajuda de uma árvore de sufixos. Tal árvore é uma versão especializada de uma \textit{Radix Tree} contendo todos os sufixos de cada string e algum marcador referente a string original.

        Com essa representação, o problema da maior substring comum se transforma em um problema de encontrar o ancestral comum mais recente entre duas folhas de origens distintas na árvore. Esse ancestral comum representa o maior prefixo comum entre todos os sufixos das duas strings, sendo portanto a maior substring comum entre elas.

        A árvore de sufixos pode ser adaptada também para manter substrings de múltiplas string distintas, permitindo a implementação eficiente da maior substring comum em uma coleção de strings. Esse adaptação não foi implementada nesse trabalho.
