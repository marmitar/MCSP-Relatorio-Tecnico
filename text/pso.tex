Utilizando a representação por grafo desenvolvida, optamos por implementar o \textit{Particle Swarm Optimization} (PSO). Esse algoritmos baseia-se no comportamento de enxames e bandos de aves -- mas também de cardumes ou até grupos humanos -- durante uma busca \cite[p.~7]{yang_nature-inspired_2010}. Cada agente busca localmente em seus arredores por posições de qualidade, participando também da comunicação do grupo para que todos saibam qual é a melhor encontrada até o momento. O PSO ganhou popularidade nas últimas duas décadas devido ao agradável balanço que proporciona entre a eficiência da busca por soluções e a facilidade de implementação e de adaptação ao problema em que é aplicado \cite[p.~640]{marti_handbook_2018}.

De forma mais prática, o algoritmo consiste em um conjunto de partículas em que cada uma possui uma posição atual e mantém gravada a melhor posição que encontrou até o momento. O enxame, que contém as partículas, também mantém a melhor posição global encontrada. A cada iteração, a posição de uma partícula é atualizada tradicionalmente utilizando 3 componentes: um aleatório, um em direção à sua melhor posição e um em direção à melhor posição global \cite[p.~642]{marti_handbook_2018}.
