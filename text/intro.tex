Diversos problemas de comparar strings, encontrando o menor número de operações necessárias para formar uma string a partir de outra, têm diversas aplicações em biologia computacional, em processamento de texto e em compressão de arquivos \cite{goldstein_minimum_2005}. O problema da partição comum mínima de strings (MCSP) se encaixa nessa classe de problemas, especificamente para o caso em que a única operação disponível é a reordenação de substrings. 

\begin{definition}
    Duas strings S e P são ditas \textbf{balanceadas} se o número de ocorrências dos caracteres de S for igual ao dos caracteres de P. 
\end{definition}

\begin{definition}
    Duas sequências de strings $\S$ e $\P$ formam uma \textbf{partição} de duas strings balanceadas $S$ e $P$ se: \cite[p.~69]{siqueira_heuristicas_2022}
    \begin{itemize}
        \item $|\S| = |\P|$;
        \item $S$ é o resultado da concatenação dos elementos de $\S$;
        \item $P$ é o resultado da concatenação dos elementos de $\P$;
        \item $\S$ é uma permutação dos elementos de $\P$.
    \end{itemize}
\end{definition}

O MCSP consiste em encontrar uma partição de tamanho mínimo (i.e. com o menor número de blocos) para duas strings balanceadas.

Ao abordar o problema pela primeira vez, alguém poderia considerar uma solução direta: testar, para todas as possíveis configurações de separação em blocos de S, se é possível formar P a partir de uma permutação e, com isso, escolher as soluções de menor tamanho. É fácil notar, no entanto, que tal algoritmo não é polinomial, de forma que é inviável para obter uma solução do problema, mesmo com instâncias de menos de 50 caracteres. De fato, provou-se que o MCSP é um problema NP-Difícil, exceto para o caso em que cada caractere ocorre apenas uma vez em cada string \cite{goldstein_minimum_2005}. Sendo assim, heurísticas são ótimos instrumentos para encontrar soluções boas para instâncias do problema.
